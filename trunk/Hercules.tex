\documentclass[a4paper,10pt,titlepage]{article}
\usepackage[left=2cm,top=3cm,right=2cm,bottom=1cm,head=2.0cm,includefoot]{geometry}
\usepackage[utf8]{inputenc}
\usepackage[spanish]{babel}
\usepackage{comment}
\usepackage{fancyhdr}
\usepackage[T1]{fontenc}
\usepackage{graphicx}
\usepackage{bookman}
\usepackage{amsmath}
\usepackage{color}
\usepackage{listings}
\usepackage{longtable}
\usepackage{moreverb}
\usepackage{booktabs}
\usepackage{multirow}
\usepackage{ulem}
\usepackage[pdfborder={0 0 0 0}]{hyperref}
\usepackage{fixltx2e}
\usepackage{float}
\usepackage{wrapfig}
\usepackage{soul}
\usepackage{t1enc}
\usepackage{textcomp}
\usepackage{marvosym}
\usepackage{wasysym}
\usepackage{latexsym}
\usepackage{amssymb}
\usepackage{microtype}
\usepackage{hyperref}
\usepackage{pdfpages} % to import PDF pages
\tolerance=1000


\title{71.12 - Estructura de las Organizaciones}
\author{\textbf{Grupo B1}}

\begin{document}

\pagestyle{fancy}
\chead{Grupo B1}
\lhead{\includegraphics[width=1.7cm]{./logo1.png}}
\lfoot{71.12 - Estructura de las Organizaciones}
\rfoot{$1^{er}$ Cuatrimestre 2011}

%documento

\section{Enunciado}

\section{Interpretaci\'on}

\title \textbf{\underline{Elevadores Hércules S.A.}}

\begin{itemize}
 \item \textbf{Resumen}\\
      Elevadores Hércules S.A. que en sus principios era una oficina contratista ubicada en Buenos Aires tuvo tal desarrollo que se convirtió 
      en una importante fábrica de ascensores. Con el pasar de los años, la demanda de ascensores fue creciendo como resultado del aumento de 
      la construcción de edificios. Los pedidos de los clientes llegaron a sobrepasar la capacidad de producción de la fábrica trayendo como 
      consecuencia el retraso de la entrega del producto y, por ende, el descontento de los clientes. Además, se sumaba que la forma de 
      organizarse de la empresa, que les había funcionado tan bien cuando había una baja demanda, no lograba adaptarse a los nuevos cambios
      empeorando aún más la situación. Se recurrió a una consultora para que estudiara la situación y encontrara una solución a los problemas.
 \item \textbf{Problemas}\\
      Los problemas que se encontraron en este caso son los siguientes:
      \begin{enumerate}
	\item La capacidad productiva no llega a satisfacer la demanda de los clientes. El departamento de ventas se comprometía a entregar el 
	producto en un lapso de tiempo que no podía cumplirse por parte del departamento de producción. Esto provocaba que ambos departamentos
	entren en conflicto.
	\item La fabricación de los ascensores no es Standard. Sólo una pequeña cantidad de piezas eran comunes a todos. El resto de las piezas 
	que componían al ascensor dependía de los requerimientos dados por el cliente.
	\item La falta de organización dentro de la misma empresa. Esto causaba que los planeadores y los jefes de sección no se enteraran cuando 
	un edificio tenía sus obras paradas haciendo que fuese mantenido el stock de sus correspondientes semielaborados. Este era un problema grave 
	ya que se podía aprovechar esos acontecimientos para avanzar en la construcción de ascensores de otros clientes.
      \end{enumerate}

\end{itemize}

\end{document}
