\documentclass[a4paper,10pt,titlepage]{article}
\usepackage[paperwidth=190mm,paperheight=290mm,left=1.5cm,top=3cm,right=0.5cm,bottom=1cm,head=2.0cm,includefoot]{geometry}
\usepackage[a4,frame,center,noinfo,horigin=-0.75in]{crop}
\usepackage[utf8]{inputenc}
\usepackage[spanish,activeacute]{babel}
\usepackage{lastpage}
\usepackage{comment}
\usepackage{fancyhdr}
\usepackage[T1]{fontenc}
\usepackage{graphicx}
\usepackage{bookman}
\usepackage{amsmath}
\usepackage{color}
\usepackage{longtable}
\usepackage{moreverb}
\usepackage{booktabs}
\usepackage{multirow}
\usepackage{ulem}
\usepackage[pdfborder={0 0 0 0}]{hyperref}
\usepackage{fixltx2e}
\usepackage{array}
\usepackage{float}
\usepackage{wrapfig}
\usepackage{soul}
\usepackage{t1enc}
\usepackage{textcomp}
\usepackage{marvosym}
\usepackage{latexsym}
\usepackage{amssymb}
\usepackage{hyperref}
\usepackage{slashbox} %slash para la tabla
\usepackage{colortbl} %tablas con colores
\usepackage{pdfpages} % to import PDF pages
\usepackage{xcolor}
\usepackage{titlesec}
\usepackage{rotating}

\renewcommand{\headrulewidth}{1pt}
\renewcommand{\footrulewidth}{1pt}

\titleformat{\section}[hang]{\Huge }{}{0px}{\centering \hspace{1ex}}{}

\begin{document}

\pagestyle{fancy}
\chead{Ing. Barmack\\Grupo B1}
\lhead{\includegraphics[width=1.7cm]{./logo1.png}}
\lfoot{71.12 Estructura de las Organizaciones}
\rfoot{$1^{er}$ Cuatrimestre 2011}
\cfoot{ P\'agina \thepage \hspace{0.5pt} de \pageref{LastPage}}

\tableofcontents
\newpage

%documento

%\part{Casos}
\vspace*{\fill}
\begin{center}
\begingroup
\titlerule
\vspace{1cm}
\section{Casos}
\vspace{1cm}
\titlerule
\endgroup
\end{center}
\vspace*{\fill}

%\section{Casos}
\subsection{Elevadores H\'ercules}
\vspace{0.5cm}
\subsubsection{Enunciado}

		\indent Elevadores H\'ercules S.A., establecida en Buenos Aires en 1919 como una oficina de contratistas, se desarrollo al punto de transformarse en una de las compa\~n\'ias m\'as importantes del mundo. En 1966, la compa\~n\'ia produc\'ia 1650 elevadores y en 1974 lleg\'o a 7.850 unidades, inclusive escaleras mec\'anicas. Aunque su planta principal est\'a ubicada en Buenos Aires, tiene oficinas comerciales en las 18 ciudades m\'as importantes del pa\'is participando con m\'as del 60\% del mercado nacional. A partir de 1970 el n\'umero de edificios comenz\'o a aumentar considerablemente. Los pedidos de los clientes tend\'ian a alcanzar l\'imites que sobrepasaban la capacidad de producci\'on de la f\'abrica. Los atrasos en la entrega de pedidos llegaron al punto de provocar serios conflictos entre los departamentos de ventas y producci\'on.\\
		\indent En funci\'on de lo anterior, la alta direcci\'on de la compa\~n\'ia decidi\'o perfeccionar el sistema de planeamiento y control de la f\'abrica.\\ \\
 
	\textbf{1 - Principales caracter\'isticas del sistema de producci\'on:}\\\\
		\indent La producci\'on de elevadores requiere cerca de 6.000 diferentes grupos de piezas de varios tipos o medidas y aproximadamente 12.000 \'items de stock. La mayor\'ia de los fabricantes depende de sus proveedores para piezas especializadas como por ejemplo motores el\'ectricos, cabinas, relees de contacto, gu\'ias, puertas metalizas y cerraduras. Al contrario de esto, elevadores H\'ercules S.A. tiene la directriz de ser autosuficiente y producir todas las piezas que utiliza. De esto resulta que la empresa tiene una producci\'on bastante diversificada, que no es com\'un en su ramo y que da origen a un complejo sistema de planeamiento y control de la producci\'on.\\
		\indent La producci\'on de elevadores no puede seguir un plan general, por que los pedidos var\'ian considerablemente de acuerdo a las necesidades de los edificios en construcci\'on. Apenas algunas partes de los elevadores H\'ercules son Standard y producidas para stock, como por ejemplo: correderas-gu\'ias, gu\'ias de puerta, cerradores, motores y conjuntos de motores generadores, relees de contacto y botones de llamada. El planeamiento de producci\'on esta dificultado tambi\'en por el desarrollo tecnol\'ogico de la construcci\'on de diferentes tipos de lugares, dependiendo por eso de condiciones que dif\'icilmente se pueden prever.\\
		\indent El equipo de producci\'on y montaje de elevadores estaba dividido en 4 grupos generales, de acuerdo con la secuencia a ser seguida en la entrega de partes, conforme al siguiente esquema:\\
		\begin{itemize}
		  \item[-] GRUPO \#1\\
		  Modelo soporte para la cabina, gu\'ias, correderas, barras, amortiguadores, base,
		  maquina y polea de desvi\'o.
		  \item[-] GRUPO \#2\\
		  Tablero de comando
		  \item[-] GRUPO \#3\\
		  Armaz\'on de cabina, Contrapesos, paragolpes, plataforma, cabina y cables de acero.
		  \item[-] GRUPO \#4\\
		  Puertas de lobby, visores, cerraduras, botones de llamada y otros detalles necesarios
		  para que complete el montaje en el edificio.
		\end{itemize}

		\indent La producci\'on de la f\'abrica estaba organizada a trav\'es de las siguientes secciones:
		\begin{enumerate}
		 \item Maquinas operativas, tornos, plegadoras, perforadoras, rectificadoras
		 \item Estampado
		 \item Montaje de maquinas
		 \item Montaje de motores
		 \item Montaje de aparatos el\'ectricos
		 \item Montaje y conexi\'on de cuadros de comando
		 \item Carpinter\'ia, fabricaci\'on de contrapesos, cabinas y puertas de acero.
		 \item Carpintero, cabinas, puertas y plataformas de madera
		 \item Pintura y galvanoplastia
		\end{enumerate}

		\indent En 1970, el planeamiento de producci\'on de elevadores H\'ercules S.A. era un simple proceso basado en reportes mensuales de campo del departamento t\'ecnico, encargado del montaje de los elevadores, formado por varios grupos de empleados especializados. Cada grupo era responsable por el control de una cierta \'area de la ciudad. El jefe de grupo visitaba peri\'odicamente a varios clientes de su localidad y estimaba futuras necesidades. Completaba un formulario de ``avances del mes'' donde volcaba los avances de cada obra indicando el grado de avance de la construcci\'on y estableciendo los programas de entrega de acuerdo con los cuatro grupos generales del proceso de producci\'on y montaje ya mencionados. Una vez que el formulario se completaba, le era entregado al planeador de la producci\'on, un antiguo supervisor que, en 1942, se convirti\'o en asistente del departamento de producci\'on a fin de controlar el proceso de planeamiento de la compa\~n\'ia.\\
		\indent A partir de los formularios de ``avances del mes'' recibidos por todas las \'areas, el planeador elaboraba el programa de producci\'on para todas las partes a ser producidas de acuerdo a la secuencia num\'erica indicada por el departamento de ventas y que obedec\'ia al orden de entrada de los pedidos de los clientes. El planeador recib\'ia tambi\'en las copias de ``orden de fabricaci\'on individual'' realizadas por el departamento de ingenier\'ia, conteniendo las especificaciones necesarias para producir cada elevador.\\ 
		\indent En la \'epoca en que la cantidad de elevadores producidos era relativamente baja en relaci\'on con la capacidad de producci\'on de la fabrica, el sistema de planeamiento descrito, probo ser simple y eficiente y pod\'ia ser f\'acilmente controlado por el planeador y por los jefes de secci\'on que en conjunto programaban la producci\'on, determinando cantidades y especificaciones, pidiendo materiales a ser producidos por la fundici\'on, de oficinas o del pa\~nol.\\
		\indent Los reportes mensuales de los grupos de campo eran suficientes para dar al planeador las informaciones en cuanto a las necesidades futuras de los edificios en construcci\'on y por lo tanto, esclarecer las prioridades de producci\'on.\\
		\indent Entretanto a partir de 1970, el n\'umero de construcciones comenz\'o a aumentar. Los retrasos en las entregas de elevadores hicieron que los jefes de campo fijasen los plazos de entrega muy anticipados en sus informes mensuales. Con eso las informaciones recibidas por el programador, fueron perdiendo parte de su valor como base para la programaci\'on. Ocurri\'o tambi\'en que ni el planeador ni los jefes de secci\'on de producci\'on eran avisados cuando un edificio ten\'ia sus obras paralizadas, haciendo que fuese mantenido el stock de sus correspondientes semielaborados. Este desperdicio agravaba todav\'ia mas la situaci\'on de los atrasos provocando graves reclamos por parte de otros clientes. Teniendo eso en vista, el departamento de ventas comenz\'o a sugerir alteraciones en las prioridades distintas a las ordenes de producci\'on, lo que llevo a los empleados a abandonar los m\'etodos de programaci\'on que hasta entonces hab\'ia sido establecidos por los jefes de grupo, pasando entonces a trabajar de acuerdo a las ordenes de ventas del departamento respectivo.\\\\

	\textbf{2 - Decisiones}\\\\
		\indent En vista de la situaci\'on, la alta direcci\'on decidi\'o perfeccionar el sistema de planeamiento y control de la f\'abrica. Contratar una consultora para que analice el caso y revertir la situaci\'on de esta compa\~n\'ia.\\ 
\newpage
\subsubsection{Organigrama Derivado del Enunciado}
  \begin{center}
  \includegraphics[width=150mm]{./herculesBN.png}
  \end{center}
\newpage

\subsubsection{An\'alisis del Caso}

		\indent Elevadores H\'ercules S.A. que en sus principios era una oficina contratista ubicada en Buenos Aires tuvo tal desarrollo que se convirti\'o        en una importante f\'abrica de ascensores. Con el pasar de los a\~nos, la demanda de ascensores fue creciendo como resultado del aumento de       la construcci\'on de edificios. Los pedidos de los clientes llegaron a sobrepasar la capacidad de producci\'on de la f\'abrica trayendo como       consecuencia el retraso de la entrega del producto y, por ende, el descontento de los clientes. Adem\'as, se sumaba que la forma de       organizarse de la empresa, que les hab\'ia funcionado tan bien cuando hab\'ia una baja demanda, no lograba adaptarse a los nuevos cambios      empeorando a\'un m\'as la situaci\'on. Se recurri\'o a una consultora para que estudiara la situaci\'on y encontrara una soluci\'on a los problemas. \\ \\

	\textbf{Aspectos Desfavorables}\\\\
		\indent Los problemas que se encontraron en este caso son los siguientes:
		\begin{enumerate}
		  \item La capacidad productiva no llega a satisfacer la demanda de los clientes. El departamento de ventas se compromet\'ia a entregar el 
		  producto en un lapso de tiempo que no pod\'ia cumplirse por parte del departamento de producci\'on. Esto provocaba que ambos departamentos
		  entren en conflicto.
		  \item La fabricaci\'on de los ascensores no es Standard. S\'olo una peque\~na cantidad de piezas eran comunes a todos. El resto de las piezas 
		  que compon\'ian al ascensor depend\'ia de los requerimientos dados por el cliente.
		  \item La falta de organizaci\'on dentro de la misma empresa. Esto causaba que los planeadores y los jefes de secci\'on no se enteraran cuando 
		  un edificio ten\'ia sus obras paradas haciendo que fuese mantenido el stock de sus correspondientes semielaborados. Este era un problema grave 
		  ya que se pod\'ia aprovechar esos acontecimientos para avanzar en la construcci\'on de ascensores de otros clientes.\\
		\end{enumerate}

	\textbf{Soluci\'on Propuesta}\\\\
		\indent Uno de los principales inconvenientes de la empresa es su diversificada producci\'on, debido a que cada elevador era fabricado siguiendo los requisitos del cliente.\\
Una soluci\'on a este problema, es la estandarizaci\'on de cierta porci\'on de la producci\'on, determinando cuales son las caracter\'isticas comunes  de los elevadores que se venden.
De esta manera, se tendr\'a un porcentaje importante de la producci\'on ya empezada y se podr\'a  terminar el producto a gusto del cliente y en el tiempo establecido. Tambi\'en se lograr\'ia una reducci\'on en los costos de producci\'on y por ende, se puede reducir los precios de ventas, mejorando la competencia en el mercado.\\ \\
		\indent Por otro lado, debido a los retrasos en las entregas, se fijaban plazos de entrega muy anticipados, provocando conflictos entre los departamentos de ventas y producci\'on. Esto se debe a la falta de comunicaci\'on entre ambos departamentos. Otro inconveniente es que ni el planeador ni los jefes de la secci\'on de producci\'on eran avisados cuando un edificio ten\'ia sus obras paralizadas, haciendo que el stock de semielaborados fuese mantenido.\\
Una soluci\'on a estos problemas, es la incorporaci\'on de un nuevo Departamento de Sistemas, que se encargar\'a de mantener actualizado el software que la empresa utilizar\'a. Este nuevo departamento recibir\'a informes de los departamentos de Producci\'on y T\'ecnico. Al recibir los informes de producci\'on, este actualizar\'a el sistema, y de esta manera el departamento de Ventas no fijar\'a fechas ficticias, sino que de acuerdo a como va la producci\'on, se podr\'an fijar las fechas correctamente, evitando los retrasos de entrega. Y al recibir los informes del Departamento T\'ecnico, el \'area de Planeamiento y Control de la Producci\'on sabr\'a cuando un edificio tiene las obras paralizadas. Y de esa manera se podr\'a mantener actualizado el stock de los semielaborados, y aprovechar los tiempos para focalizarse en la terminaci\'on de otros pedidos.
La creaci\'on de los programas podr\'ia ser una actividad tercerizada ya que no se prev\'e que estos est\'en cambiando constantemente.


\newpage
	\textbf{Organigrama de la solución planteada}\\\\
    \begin{center}
    \includegraphics[width=90mm]{./herculesBNSol.png}
    \end{center}

%%%%%%%%%%%%%%%%%%%%%%%%%%%%%%%%%%%%%%%%%%%%%%%%%%%%%%%%%%%%%%%%%%%%%%%%%%%%%%%%%%%%%%%%%%%%%%%%%%%%%%%


\newpage
\subsection{Tejedur\'ias Dolly}

\vspace{0.5cm}

  \subsubsection{Enunciado}
	\vspace{0.5cm}
	 \textbf{1- Introducci\'on}\\\\
		\indent Este ejemplo ha sido tomado de un caso real, al cual se le han practicado algunas simplificaciones para convertirlo en un caso al que se le pueden aplicar las t\'ecnicas de organizaci\'on.\\
		\indent Tejedur\'ia DOLLY es una empresa dedicada a la fabricaci\'on de prendas de tejido de punto partiendo de hilados de fibras naturales, artificiales y/o sus mezclas.\\
		\indent Es una empresa Pyme caracter\'istica de nuestro pa\'is, en la cual se ha puesto m\'as voluntarismo que aplicaci\'on de m\'etodos cient\'ificos o profesionalizados.\\
		\indent La empresa es del tipo familiar, con conformaci\'on legal de Sociedad An\'onima, en la cual el \'unico director, que trabaja como gerente operativo, es un profesional de Ingenier\'ia, y ha dirigido esta empresa en los \'ultimos 6 años.\\
		\indent La empresa comercializa sus productos en el mercado local, el 90\% a trav\'es de 10 negocios propios, y un 10\% a trav\'es de distribuidores o negocios que compran directamente en el deposito.\\
		\indent Los diez locales est\'an distribuidos en Capital, Gran Buenos Aires, La Plata, C\'ordoba, Mar del Plata y Mendoza. Los negocios tienen de dos a cuatro empleadas, de las cuales, una de ellas, es la encargada del negocio. En total el plantel de vendedoras es de 45 personas.\\
		\indent Hay un jefe de ventas que supervisa, la venta de los locales, la venta mayorista y tambi\'en tiene a su cargo el dep\'osito de productos terminados.\\
		\indent La empresa cuenta con un sistema inform\'atico que interconecta los locales con la fabrica, que por ser un sistema un tanto r\'igido sufre a menudo problemas que impiden que la informaci\'on sobre ventas, stock y pedidos de los clientes se actualicen correctamente.\\
		\indent Esto genera inconvenientes en las tiendas, que pueden quedar desabastecidas o sobrestockeadas, adem\'as de generar inconvenientes contables.\\
		\indent Las ventajas competitivas de la empresa son trabajar con materia prima de calidad, en este caso importada de una importante firma Italiana, manteniendo un contrato de exclusividad para la regi\'on, esto otorga una diferenciaci\'on en los productos finales.\\
		\indent La segunda ventaja es ser una fabrica de tejido de punto integrada verticalmente hasta llegar al consumidos, caracter\'istica que es poco com\'un.\\
		\indent Dolly fabrica prendas con alto contenido de diseño con lo que ha logrado imponerse en un mercado dif\'icil. Nuestro an\'alisis se realiza a mediados del año 2000 donde lo \'unico que pod\'iamos asegurar es que el tan pronosticado Y2K fueron solo amenazas.\\
		\indent Al ser prendas de diseño se realizan en cantidades chicas 500 a 2000 unidades con excepci\'on de algunos comoditis, que se hacen de a 5000 unidades y se repiten producciones.\\ 
		\indent Deben agregar a esto que cada partida se tiñen de 4 o 5 colores diferentes. Todos los años, el diseñador y el dueño de la empresa viajan a Europa para relevar las tendencias est\'eticas, en modelos y colores dado que estando en contraestaci\'on, esto le permite definir anticipadamente la producci\'on de cada temporada. Esto, agregado a la relaci\'on precio-calidad, resultante de su m\'etodo de comercializaci\'on pone a la empresa entre las m\'as importantes del mercado.\\
		\indent La empresa tiene un negocio donde vende mercader\'ia de diseños exclusivos y muy cuidada calidad orientado a boutiques de alta costura, para lo cual cada temporada, organiza un desfile con su correspondiente organizaci\'on. El resto de los negocios venden productos con un diseño Standard de acuerdo a las necesidades del mercado al cual va dirigido.\\
		\indent En este caso tambi\'en tienen muy buena acogida, logrando vender importantes cantidades y con un nivel de ventas que si bien tiene picos se mantienen todo el año manteniendo la operaci\'on de los locales a pleno.\\
		\indent La empresa, en vista del \'exito comercial que ten\'ia, y esperando mejorar la calidad y la productividad, en el año 1998 compr\'o maquinas de una nueva tecnolog\'ia para reemplazar las antiguas maquinas. Estas maquinas producen prendas con la forma exacta que debe tener cada pieza, evitando la operaci\'on de corte y todo el desperdicio de material que se produce al cortar con los moldes los paños de tejido.\\
		\indent Desgraciadamente no se tomaron las precauciones de tomar personal capacitado para el manejo de esas maquinas (o capacitar personal propio) y estas no solo no lograban la productividad Standard sino que exist\'ian muchos desperdicios de puesta a punto o de programas mal desarrollados. Tambi\'en se produc\'ian problemas de barrado (que es una l\'inea en el tejido debido a diferencia de tensi\'on).\\
		\indent Si bien se aumento la producci\'on total con estas maquinas, no fue en la cantidad que se esperaba y el 70\% del tejido era realizada por las viejas maquinas circulares.\\
		\indent Estas m\'aquinas adem\'as como fueron compradas con credito a 5 años produc\'ian una erogaci\'on mensual mayor que la ganancia que se produc\'ia por su uso.\\
		\indent La f\'abrica se encuentra en el gran Buenos Aires, y desarrolla sus actividades en un edificio alquilado de 4000 m2. Compra hilado de pelo (un tipo de lana) y algod\'on seg\'un la temporada, el 60\% de esta materia prima es importada de Italia. El hilado es la principal de las materias primas, al que le corresponde un porcentaje importante del total de las compras.\\ 
		\indent A continuaci\'on se hace un a breve descripci\'on de la empresa y su proceso productivo.\\

	\vspace{0.5cm}
	 \textbf{2- Relevamiento del Proceso Productivo}\\ \\
		\indent El proceso productivo est\'a dividido en siete sectores consecutivos a trav\'es de los cuales se va transformando la materia prima hasta obtener el producto terminado, embalado y puesto a disposici\'on del dep\'osito de productos terminados. La distribuci\'on en planta es por procesos y los sectores son:\\
			\begin{enumerate}
			 \item Dep\'osito de materias primas.
			 \item Tejedur\'ia: 
				 \begin{itemize}
				 \item Enconado
 				 \item Tejido
 				 \item Lavado
				 \end{itemize}
 			 \item Corte
 			 \item Confecci\'on
 			 \item Tintorer\'ia
 			 \item Terminaci\'on
 			 \item Expedici\'on
			 \end{enumerate}

		\indent En algunos casos, estos procesos son llevados a cabo a trav\'es de terceros, seg\'un su naturaleza y los planes de producci\'on. Las empresas proveedoras de estos servicios son llamadas ``fasones'' o ``terceros''.\\
		\indent El proceso de ``Tintorer\'ia'' que en realidad es el teñido de las prendas se realiza cuando corresponde, y es siempre tercerizado puesto que TEJEDURIA DOLLY no posee instalaciones para tal fin.\\ \\
			\begin{center}\textbf{\underline{Descripci\'on de cada proceso}}\end{center}
			\begin{enumerate}
			\item \textbf{\underline{Dep\'osito de MP}}
			\item  \textbf{\underline{Tejedur\'ia:}}\\\\
			\indent El sector de tejedur\'ia se compone de tres partes\\
				\begin{itemize}
				\item \underline{Enconado:} Los conos de hilado se re-enconan en casi todos los casos con el fin de lograr una tensi\'on pareja y acorde a los requerimientos de cada telar. En el mismo proceso se eliminan nudos y aglomeraciones de fibras y se le aplica parafina s\'olida al hilado, para facilitar su tejido.
				\item \underline{Tejedur\'ia:} Este es el proceso fundamental y caracter\'istico de la empresa. Se teje en los telares seg\'un especifican las \'ordenes de producci\'on y las fichas de producto.\\
				\indent Las partidas que no contin\'uen su proceso en el sector de lavado, se someten aqu\'i a un control estad\'istico de calidad.\\
				\item \underline{Lavadero:} Mediante este proceso se lavan las piezas tejidas en una soluci\'on acuosa de distintos agentes detergentes y humectantes. Luego se las centrifuga y finalmente se las seca con aire caliente. La finalidad de este proceso es fundamentalmente la de lograr una estabilidad dimensional y una eliminaci\'on de las tensiones internas adquiridas durante el tejido.\\ 
				\indent Las partidas listas para continuar el proceso son sometidas aqu\'i a un control estad\'istico de calidad.\\
				\end{itemize}
			\item \textbf{\underline{Corte:}}\\\\
			\indent Para la obtenci\'on de diversas formas a partir de piezas de forma rectangular, se recortan los paños tejidos seg\'un las formas y dimensiones especificadas en las planillas de producto y el encogimiento previsto durante el proceso de teñido. Luego de cortadas, las piezas se ordenan por tipo de pieza, se atan y se identifican claramente los bultos. Las partidas ya cortadas se estiban en las estanter\'ias si van a ser confeccionadas por el sector de confecci\'on y se embolsan si es que van a ser confeccionadas por talleres de terceros. Estas bolsas se apilan en forma separada.\\
			\item \textbf{\underline{Confecci\'on:}}\\\\
			\indent Las distintas piezas tejidas que componen una prenda se unen aqu\'i mediante distintos tipos de costura. Esto se hace siguiendo las reglas de arte en la materia, y en funci\'on de las especificaciones consignadas en las fichas de cada producto. Para la confecci\'on interna, la persona encargada del sector determina en funci\'on de la Ficha de Producto, que m\'aquina corresponde utilizar para cada una de las costuras necesarias y que secuencia se seguir\'a hasta confeccionar todas las prendas de la partida. La misma encargada se ocupa de la traslaci\'on de los bultos entre los diferentes puestos de trabajo, y de la entrega de las partidas ya confeccionadas al sector de terminaci\'on.\\
			\item \textbf{\underline{Tintorer\'ia:}}\\\\
			\indent Este trabajo se terceriza, pero exige un muy prolijo conteo de expedici\'on y recepci\'on. Este es el momento en que se toma la decisi\'on respecto del color de las prendas.\\
			\indent Tareas previas al teñido:\\
			\indent \underline{ - Control de colores:} Se controla que los colores recibidos desde tintorer\'ia se correspondan con la carta de colores prevista.\\
			\item \textbf{\underline{Terminaci\'on:}}\\\\
			\indent Cuando las partidas est\'an confeccionadas, la persona encargada del sector de terminaci\'on determina la secuencia de tareas a realizar sobre la prenda a los efectos de dejarla lista para la venta. La siguiente es la secuencia general:\\

			\item \textbf{\underline{Expedici\'on:}}\\\\
			\indent En este sector se almacenan las prendas para ser enviados en el momento oportuno.\\
			\end{enumerate}
\newpage
	 \textbf{3- Detalles del Personal}\\ \\
			\indent El sector fabril se encuentra distribuido en las secciones anteriores que detallaremos.\\
			\indent En RECEPCIÓN DE MATERIAS PRIMAS Y OFICINA DE PRODUCCIÓN se trabaja un turno con dos personas.\\
			\indent En TEJEDURÍA existen dos encargados que supervisan el sector de ENCONADO con dos operarios, TEJEDURÍA tiene 18 operarios y el LAVADERO que cuenta con 4 personas. Estos sectores trabajan dos turnos entre los que se divide el personal. El sector de CORTE tiene un encargado con cuatro personas a cargo y CONFECCIÓN tiene un encargado, un ayudante y 18 operarias.\\ 
			\indent Por \'ultimo la encargada de TERMINACIÓN tiene 14 operarias. De este sector la mercader\'ia pasa a DEPOSITO que cuenta con un encargado y dos personas. Aunque este ultimo sector depende del jefe de ventas.\\
			\indent La DIRECCIÓN DE FABRICA est\'a a cargo de un jefe de producci\'on y cuenta como auxiliares con una persona de mantenimiento y un chofer.\\
			\indent Existe un sector de INGENIERÍA Y CALIDAD con un encargado y dos personas a su cargo.\\
			\indent Un JEFE DE OPERACIONES y producto con una encargada de medidas y puesta a punto.\\
			\indent La ADMINISTRACIÓN esta dirigida por un contador y se maneja con cuatro personas, que tambi\'en asisten al gerente.\\
			\indent El personal de la f\'abrica eran unas 80 personas, m\'as el personal de los locales (50 empleados) y administrativos, de log\'istica y directivos agregaban 20 personas. Ten\'ian una antigüedad promedio de 10 años y conoc\'ian bien su oficio, pero se resist\'ian a los cambios tecnol\'ogicos.\\\\
			\underline{Incorporaci\'on de sector}\\\\
			\indent En este momento (año 2000) se dispuso incorporar un sector de calidad con el fin de reducir el desperdicio que rondaba el 20\% (adicional a lo que se explico de los telares computarizados. Pero esta tarea tampoco fue f\'acil debido a que en la industria textil se acostumbra a tener medidas aproximadas y el personal reclutado no tenia la experiencia en el rubro como para liderar este cambio.\\ \\
			\indent \underline{Se pide} que planteen una estructura adecuada alternativa a la presentada y las medidas que se deben poner en practica para mejorar la situaci\'on problem\'atica en que se encuentra la empresa.\\

\begin{center}
\includegraphics[scale=0.75]{./DollyEsquemaProduccion.png}
\end{center}


\subsubsection{Organigrama Derivado del Enunciado}
	\begin{center}
	\includegraphics[width=450pt]{./DollyOrganigrama.png}
	\end{center}


	\vspace{0.5cm}
\subsubsection{An\'{a}lisis del caso}

		\indent Tejedur\'{i}a Dolly S. A. es una empresa Pyme de tipo familiar, con f\'{a}brica en Buenos Aires y locales en Capital, Gran Buenos Aires, La Plata, C\'{o}rdoba, Mar del Plata y Mendoza. Su director es un profesional de Ingenier\'{i}a.\\
		\indent Dentro del rubro textil, se dedica a la fabricaci\'{o}n de prendas de tejido de punto a partir de hilados de fibras naturales, artificiales o bien sus mezclas. Comercializa sus productos en el mercado local, en su mayor\'{i}a a trav\'{e}s de negocios propios (90\%) pero tambi\'{e}n mediante distribuidores y negocios que compran directamente en el dep\'{o}sito.\\
		
		\textbf {Ventajas}\\\\
		\indent Dentro de las principales ventajas que tiene la empresa, se puede hacer menci\'on de:
		\begin{enumerate}
		 \item Al trabajar con materia prima de calidad, en este caso importada de una importante firma italiana (con un contrato de exclusividad para la regi\'{o}n), otorga una diferencia en los productos finales.
		 \item Tiene la caracter\'{i}stica de ser una f\'{a}brica de tejido de punto integrada verticalmente hasta llegar al consumidor.
		 \item El nivel de ventas, si bien tiene picos, mantiene todo el a\~no la operaci\'{o}n de los locales a pleno. Un factor, que influye positivamente en esto, es la adaptaci\'{o}n que se hace sobre las prendas para lograr su adecuaci\'{o}n a las tendencias est\'{e}ticas del momento, en cuanto a modelos y colores. Por tanto, consecuencia de su m\'{e}todo de comercializaci\'{o}n es que la relaci\'{o}n precio-calidad ponga a la empresa entre las m\'{a}s importantes del mercado.
		 \item Cuenta con locales propios donde puede ofrecer su amplia variedad de productos. La empresa comercializa un 90\% de sus productos en estos locales, es decir, que es un factor de peso. Además, tener locales propios le permite tener su propio sistema de ventas y promocionar sus nuevos lanzamientos.
		 \item La empresa logr\'o posicionarse entre las mejores en su negocio. Lo cual mejora la imagen de la misma. Cuenta con ventajas distintivas para hacerse resaltar, y ha logrado mantener un alto nivel de producci\'on que mantiene un nivel constante de ventas durante el a\~no.
		 \item Otro punto a favor es que posee empleados con promedio de 10 a\~nos de antig\"uedad, por lo tanto son experimentados en sus \'areas de trabajo.\\
		\end{enumerate}
		
		
		\textbf {Aspectos Desfavorables}\\\\
		\indent La empresa tiene dos principales problemas:
		\begin {enumerate}
		 \item La rigidez del sistema inform\'{a}tico que interconecta los locales con la f\'{a}brica genera retrasos y confusiones, debido a que impide la actualizaci\'{o}n inmediata de la informaci\'{o}n sobre ventas, stock y pedidos de los clientes. Esto resulta en detrimento de las tiendas, se producen circunstancias tanto de escasez como de abundancia de producto, adem\'{a}s de inconvenientes contables.
		 \item La inserci\'on de maquinarias tecnol\'ogicamente avanzadas para aumentar la cantidad y calidad de producci\'on. \'Esto tuvo el efecto contrario, ya que los operarios no estaban capacitados para dichas m\'aquinas. Si bien aument\'{o} la producci\'{o}n total, no s\'{o}lo no lleg\'{o} a alcanzarse la cantidad esperada sino que adem\'{a}s se ocasionaron desperdicios en la puesta a punto y deficiencias por programas mal desarrollados.\\
		\end {enumerate}

	\vspace{0.5cm}
\subsubsection{Soluci\'on propuesta}
		\indent En la empresa se observa una resistencia de los empleados a los cambios tecnol\'ogicos, esto es el causante de los dos grandes problemas antes mencionados. Hay que resolver esta cuesti\'on para que la empresa quede en \'optimas condiciones.\\
		\indent Es necesario poder capacitar a los empleados para que puedan aprovechar al máximo las funcionalidades de las maquinarias adquiridas. Para ello se propone que la empresa DOLLY contrate a personal externo para transmitir los conocimientos sobre como operar con la nueva tecnología. Es decir, proponemos que la empresa financie cursos para la capacitación de su personal. \\
		\indent El curso podría llevarse a cabo dentro de las instalaciones de la fábrica, ya que la misma, cuenta con las máquinas necesarias para poder realizar las prácticas y que el empleado se vaya familiarizando con la utilización de las mismas.\\
		\indent Las prácticas podrían desarrollarse fuera de la jornada laboral, para no interferir con la producción, y una vez finalizado el curso recibir alguna documentación que diga que el empleado concluyó con éxito el curso y que a adquirido la destreza necesaria para trabajar con las máquinas. Esta documentación, si bien sólo tendrá valor dentro de la empresa, marcará una distinción con los demás empleados.\\
		\indent Sumado a esto, como las prácticas son fuera de la jornada laboral, no pueden ponerse como obligatorias ya que sino deberíamos pagarle a los operarios por esas  horas en las que no trabajan, pero concurre al curso. También, si se imponen como obligatorias puede que los empleados sientan cierto rechazo a  querer participar porque está dentro de su tiempo libre. Una forma de incentivarlos a querer tomar este curso, sin necesidad de tener que obligarlos, podría ser aumentarles el sueldo a aquellos que cuenten con el certificado de haber concurrido al curso y trabajen con las máquinas nuevas.\\
		\indent Además, una vez que se cuente con una cantidad de personal capacitado, los más antiguos y con más experiencia, podrán capacitar a los otros empleados de la empresa, sin necesidad de tener que seguir contratando personas externas a la organización. También, otro beneficio respecto a que alguien que trabaja dentro de la empresa capacite a otros, es que éste hará hincapié en aquellas funcionalidades y aspectos que a la organización le interesa. \\
		\indent Para poder llevar a cabo esta idea, es necesario contar con un área de Capacitación que maneje el tema de la financiación de los cursos y, además, deberá encargarse de emitir los certificados al personal que fue capacitado y dejar una constancia de a quien fue dado dicho documento para que no pueda ser fraguado. Esta nueva área, también, deberá organizar los horarios de los cursos para que estos no interfieran con el horario en el que se están usando las máquinas para fabricar los productos que se van a vender.  Por ejemplo, si el curso necesita una máquina para realizar una práctica deberá ver cuando hay disponibilidad de las mismas.\\
		\indent Además, los cursos deberán contar con un cupo. Como son cursos prácticos más que teóricos, es aconsejable que sean cursos pequeños para que todos puedan realizar las prácticas sin inconveniente. Otro motivo para tener cupos, es que el área de Capacitación pueda determinar a quien darles la facilidad de tomar el curso y a quienes no, ya que el curso es pagado por la empresa. Por ejemplo, no sería beneficioso capacitar a una persona que le faltan pocos años para jubilarse, en cambio, si podría invertirse en la gente más joven porque es la que va a aprovechar por más tiempo estos conocimientos y, además, tienen más facilidad para aprender. \\
		\indent Por otro lado, habíamos hecho mención de un segundo inconveniente: la rigidez del sistema informático con el que se trabaja. Queda claro que el sistema no es bueno ya que le trae varios problemas como la falta y sobre stock en las tiendas. Es decir, es necesario desarrollar un nuevo sistema para poder resolver aquellos inconvenientes. También, podría contratarse personas externas a la empresa para que desarrollen un sistema que cumpla con sus necesidades y que sea más sencillo de usar. Estas mismas personas deberían capacitar a un grupo de personal de empresas DOLLY para que aprendan a utilizar el programa.\\
	\vspace{0.5cm}

\subsubsection{Matriz FODA}
	\vspace*{0.5cm}
	\begin{tabular}{|c|c|}
	\hline 
	\textbf{Fortalezas} & \textbf{Oportunidades} \\ \hline & \\
- Trabajar con materia prima de calidad. & - Avance tecnol\'{o}gico.\\ &\\
- Ser una f\'{a}brica de punto integrada & - Posicionarse mejor en el mercado\\ 
  verticalmente hasta llegar al consumidor. & debido a su alto contenido de \\ & dise\~no en sus prendas. \\
- Estudio de las tendencias de moda. & \\ & \\
- Excelente calidad de productos.& \\& \\
- Contar con locales propios, & \\ lo que permite aplicar una estrategia de& \\ mercado a nivel empresa.& \\ &\\
	\hline
	\textbf{Debilidades} & \textbf{Amenazas} \\
	\hline &\\
- Contar con un sistema inform\'{a}tico r\'{i}gido& - Entrada de nuevos competidores. \\que sufre continuos problemas que impide la &\\ actualizaci\'{o}n de la informaci\'{o}n. &\\ &\\
- Alto nivel de desperdicio en el procesamiento &\\ de la materia prima debido a los &\\ m\'{e}todos y maquinarias actuales. &\\ &\\
- Falta de capacitaci\'{o}n del personal para &\\el uso de las nuevas maquinarias. Esto &\\ produce p\'{e}rdida de rendimiento en las mismas. &\\ &\\
- Personal se resiste a cambios tecnol\'{o}gicos. &\\ &\\
	\hline    
	\end{tabular}


\subsubsection{Organigrama de la soluci\'on propuesta}
	\begin{center}
	\includegraphics[width=415pt]{./DollyOrganigramaSolucion.png}
	\end{center}

%%%%%%%%%%%%%%%%%%%%%%%%%%%%%%%%%%%%%%%%%%%%%%%%%%%%%%%%%%%%%%%%%%%%%%%%%%%%%%%%%%%%%%%%%%%%%%%%%%%%%%%

\newpage
\subsection{Emporio Automotor}
\vspace{0.5cm}
\subsubsection{Enunciado}

		\indent La empresa Emporio Automotor fue fundada en 1967 como una manufacturera y distribuidora mayorista de autopartes. Inicialmente se instalaron en un garaje. Luego, con un espacio mayor para almacenar inventario, pudieron ofrecer una l\' inea m\' as amplia de autopartes. En esa \'epoca los automovilistas de Estados Unidos comenzaron a conservar sus coches por m\'as tiempo, con lo cual, en combinaci\'on con el mayor surtido de Emporio Automotor, contribuy\'o a que el negocio tuviera una expansi\'on explosiva hacia mediados y fines del '70. Para los primeros años de los ’90, Emporio Automotor era el mayor distribuidor independiente de autopartes de la regi\'on norte de Estados Unidos.\\
		\indent El año pasado la empresa se mud\'o a un nuevo conjunto de oficinas y dep\'ositos, cubriendo algo m\'as de 30.000 metros cuadrados de superficie.\\
		\indent En la actualidad, la producci\'on se ha incrementado en un 20\% y la ocupaci\'on del dep\'osito se ha incrementado desde el 65\% en el momento de la inauguraci\'on hasta alcanzar el 90\%. En el per\'iodo transcurrido, por otro lado, el volumen de ventas no ha aumentado. El r\'apido crecimiento de los inventarios indujo entonces a la Direcci\'on a contratar a un especialista en gesti\'on de inventarios para atacar el problema.\\
		\indent La persona adecuada para esto es Susan Torio, quien es tomada como Gerente de Abastecimiento. Sue es una Ingeniera Industrial recientemente graduada que espera con ansia el momento en que habr\'a de enfrentarse por primera vez con un problema del mundo real.\\
		\indent El primer d\'ia recibe un informe sobre el estado del inventario, y de las compras pendientes. Al principio de un largo listado impreso aparece una nota manuscrita de parte del Gerente de Compras: ''Adjunto encontrar\'a los datos referentes a los niveles de inventarios y el grado de cumplimiento con los clientes. Tenga la seguridad que los niveles son precisos, porque al final de la semana pasada efectuamos un conteo f\'isico completo. Desafortunadamente, no contamos con registros compilados en algunas \'areas, pero est\'a usted en libertad de obtenerlos por s\'i misma. ¡Bienvenida a bordo!''.\\
		\indent Un poco molesta por no tener disponible toda la informaci\'on, Sue decide seleccionar al azar una pequeña muestra de aproximadamente 100 elementos, y compilar personalmente el inventario y el grado de cumplimiento con los clientes para formarse una idea del panorama general.\\
		\indent Los resultados de este experimento le revelan por qu\'e Emporio Automotor ha decidido convocarla. Parece que el inventario est\'a desperdigado en los lugares m\'as inadecuados. A pesar de que la empresa tiene unos 60 d\'ias de stock en sus inventarios, el grado de cumplimiento con los clientes no es del todo satisfactorio. \\
		\indent Sin embargo, sabe que su capacidad de introducir cambios significativos todav\'ia es limitada, a menos que sepa c\'omo generar cambios positivos de resultado comprobable de inmediato.\\

\newpage
\subsubsection {An\'alisis del Caso}
		
		\indent En principio la empresa se instal\'o como manufacturera y distribuidora mayorista de autopartes. Las demandas fueron creciendo (porque los clientes tend\'ian a tener sus autos cada vez por m\'as tiempo), y debieron implementarse algunos cambios para aumentar el espacio donde almacenar el inventario, de modo que se pudiera desarrollar una l\'inea mayor del producto. Esto deriv\'o en una r\'apida expansi\'on del negocio, y Emporio Automotor lleg\'o a convertirse en el mayor distribuidor independiente de autopartes de la regi\'on norte de Estados Unidos.\\
		\indent La empresa se traslad\'o a un nuevo conjunto de oficinas y dep\'ositos, que ocupaban m\'as de 30.000 metros cuadrados de superficie.\\
		\indent En la actualidad, la producci\'on se ha incrementado en un 20\% y la ocupaci\'on del dep\'osito del 65\% en el momento de la inauguraci\'on alcanz\'o llegar hasta el 90\%. 
		\indent En el per\'iodo transcurrido, por otro lado, el volumen de ventas no ha aumentado.\\

	\textbf{Aspectos Desfavorables}\\\\
		\indent El r\'apido crecimiento de los inventarios indujo a la Direcci\'on a buscar un especialista en gesti\'on de inventarios para atacar el problema. Se contrat\'o entonces a Susan Torio como Gerente de Abastecimiento, puesto que es funcionalmente diferente al de Gerencia de Compras.\\
		\indent Torio recibe un informe sobre el estado del inventario y las compras pendientes, pero como este resulta incompleto opta por seleccionar al azar una pequeña muestra de alrededor de cien elementos y compilar personalmente el inventario junto con el grado de cumplimiento a los clientes. El resultado le revela que, a pesar de que la empresa tiene unos sesenta d\'ias de stock en sus inventarios, el \'ultimo aspecto -el grado de cumplimiento con los clientes- no es del todo satisfactorio.\\

\newpage
\subsubsection{Resoluci\'on del Caso}

\begin{enumerate}
  \item Cu\'al es la estructura de la empresa antes de la contrataci\'on de Sue?\\
  - Desarrolle organigrama.
     \begin{center}
      \includegraphics[scale=0.55]{./orgEmporioAutomotor.png}
     \end{center}
  
  \item Cu\'al es la diferencia funcional entre una Gerencia de Compras y Una Gerencia de Abastecimiento?

\textbf{Gerencia de Compras}

	\indent El objetivo del departamento de compras es adquirir los insumos, repuestos y materiales buscando obtener los mejores precios y condiciones, calidad, financiamiento, y mantener buena relaci\'on con los proveedores. Los costos repercutir\'an directamente en el precio de venta del producto final, si estos son bajos, podr\'a ofrecerse un precio de mercado competitivo. Sus principales funciones son: 
		\begin{itemize}
		\item Dar continuidad al abastecimiento.
		\item Reducci\'on de costos y obtenci\'on de utilidades. Es decir, obtener la mejor relación entre precio-calidad para la organización.
		\item Control de convenio para obtener financiamiento.
		\item Control de tratos comerciales con los proveedores. Analizar proveedores actuales y tener una lista de proveedores alternativos.
		\item Obtenci\'on de informaci\'on y asesoramiento.
		\end{itemize}
	\indent Es aconsejable que haya un solo departamento encargado de la compra para evitar la duplicidad de compra.\\\\
\newpage

\textbf{Gerencia de Abastecimiento}

	\indent Se encarga de almacenar insumos, repuestos y materiales que se consumen dentro del establecimiento necesarios para el cumplimiento de las tareas de cada \'area. Sus principales funciones son: 
		\begin{itemize}
		\item Debe formular y aplicar una política de abastecimiento.
		\item Conocimiento de la necesidad. Determinar las cantidades necesarias.
		\item Hacer y esperar el pedido.
		\item Recibir los art\'iculos o servicios.
		\item Controlar el inventario.Es necesario contar en forma regular y sostenida con todos los elementos requeridos para la realizaci\'on del trabajo.
		\item Distribuir los productos a los departamentos.
		\end{itemize}
	\indent El \'area de abastecimiento no se encarga de la comprar directa de los insumos.\\\\
	
	\indent \textit{Bibliografía consultada: Boland,Carro, otros; \underline{Funciones de la administración 1era edición}; Editorial de la Universidad Nacional del Sur.}\\

  \item P\'ongase en el lugar de Sue: C\'omo armar\'ia la estructura de la nueva Gerencia de Abastecimiento?
  \\ \\
  Dos son los principales problemas a solucionar en el debut de la nueva Gerencia de Abastecimiento: conocer si es conveniente o contraproducente el alto nivel de stock manejado actualmente por la empresa, y lograr organizar y clasificar el stock, cuyo desorden impide la entrega en tiempo y forma a los clientes. Para ello, Sue deber\'a definir pol\'iticas claras y estrictas para el almacenamiento de autopartes, y su correspondiente clasificaci\'on y organizaci\'on en sectores. Como primer paso, deber\'a definir sub-\'areas f\'isicas en el sector de dep\'osito y elaborar un Manual de Abastecimientos que definir\'a las pol\'iticas del sector. Adem\'as, ser\'ia conveniente que llevase a cabo un estudio sobre los niveles de stock actual de cada producto y su relaci\'on con la demanda del mismo, para conocer si existe un desbalance relativo en los stocks (como por ejemplo, poco stock de un producto muy demandado y mucho stock de otro que no lo es). Realizado dicho an\'alisis, podr\'a enviar un informe al \'area de Producci\'on para reorganizar eficientemente las tareas pendientes de la misma. Independientemente de las personas que Sue pueda necesitar para ayudarla, la nueva Gerencia de Abastecimiento inicialmente ser\'a conformada alrededor de las funcionalidades o sub-\'areas de: Relevamiento y control de inventario, Elaboraci\'on de Manuales y Normas (directivas), Control de cumplimiento de normas, Elaboraci\'on de Informes para Producci\'on, An\'alisis estad\'istico de demandas y stocks.
  
  
  \item Cuales son las Fortalezas y Debilidades de Emporio Automotor?\\ 
    \\ \textbf{Fortalezas }\\
    
    \begin{enumerate}
      \item[-] La trayectoria de la empresa.
      \item[-] Ser el mayor distribuidor independiente.
      \item[-] La infraestructura de la empresa.
     \end{enumerate}  
    \textbf{Debilidades} \\ 
   \begin{enumerate}
     \item[-] Tener el inventario desperdigado en lugares inadecuados.
  
   \end{enumerate} 
\end{enumerate}
%%%%%%%%%%%%%%%%%%%%%%%%%%%%%%%%%%%%%%%%%%%%%%%%%%%%%%%%%%%%%%%%%%%%%%%%%%%%%%%%%%%%%%%%%%%%%%%%%%%%%%%

\newpage
\vspace*{\fill}
\begin{center}
\begingroup
\titlerule
\vspace{1cm}
\section{Empresa: Imagen y Comunicaci\'on}
\vspace{1cm}
\titlerule
\endgroup
\end{center}
\vspace*{\fill}

\newpage

\subsection{Tabla de Selecci\'on de Empresas}

Para elegir la empresa con la que se va a trabajar tuvimos en cuenta los siguientes criterios:
\begin{itemize}
\item Contacto: considerado uno de los elementos más importantes por eso su peso es diez. El contacto es el que nos permitirá conseguir las entrevistas que sean requeridas, preferentemente se intentará buscar que el contacto sea un familiar o conocido para tener mayor seguridad. 
\item Ubicación: la ubicación de la empresa también es un factor de gran consideración ya que será necesario recurrir a la empresa en más de una ocasión para poder entrevistar al personal. Por estos motivos, el peso es de diez.
\item Centralización: la centralización hace referencia que tan diseminada está la empresa. Es decir, si los distintos sectores productivos de la fábrica se encuentran divididos o no, si tiene locales en una zona o más bien por zonas dispersadas, etc. El peso en este caso es de ocho, se intentará buscar empresas que se encuentren centralizadas. 
\item Tamaño: para poder analizar la empresa utilizando los conceptos vistos en clase es necesario que la empresa cuenta con una cantidad de personal que ronde entre 45 a 300. El tamaño de la empresa tendrá un peso de siete.
\item Disponibilidad: este aspecto hace referencia a la disponibilidad para la entrevistas. Tiene un peso de siete.
\end{itemize}

En la siguiente tabla aparecen las empresas que conseguimos con sus respectivos puntajes. Cabe aclarar que la puntuación es subjetivo.\\
La empresa elegida es Imagen y Comunicaci\'on ya que tiene el puntaje m\'as alto y se han mostrado interesados en ayudarnos
con el trabajo de campo.
\newcolumntype{x}[1]{%
>{\centering\hspace{0pt}}p{#1}}%
\begin{table}[h]
\begin{center}
\begin{tabular}{|c|c|c|c|x{3cm}|c|}
  \hline
  \backslashbox{Caracter\'istica}{Empresa} & Peso & ServiFlex & Grupo Al Sur & Imagen y Comunicaci\'on & Punto1\\
  \hline
  Contacto 		& 10 & \textit{10*} 8 & 9 & 10 & 5\\
  \hline
  Ubicaci\'on 		& 10 & \textit{10*} 8 & 8 & 8 & 8\\
  \hline
  Centralizaci\'on 	& 8  & \textit{8*} 10 & 9 & 6 & 10\\
  \hline
  Tama\~no 		& 7  & \textit{7*} 10 & 7 & 10 & 7\\
  \hline
  Disponibilidad	& 7  & \textit{7*} 6 & 4 & 9 & 7\\
  \hline
\rowcolor[gray]{0.9} Total&- & 352 & 319 & \textbf{361} & 308\\
  \hline
\end{tabular}            
\end{center}
\caption{Puntaje de Empresas}
\end{table}

Para calcular la puntuación final, se multiplica el puntaje del criterio considerado por el peso del mismo, este paso se repite para los cinco criterios. Luego se suman los resultados parciales, obteniendo la puntuación final.


\newpage

\subsection{Minuta de Reuni\'on del d\'ia 04 de abril}
\vspace{1cm}
\begin{center}
\begin{tabular}{|c|c|c|}
	\hline
	\textbf{Fecha:} 04/04/2011 &  \textbf{Comienzo:} 14:00  &  \textbf{Fin:} 15:30 \\ \hline	

	\multicolumn{3}{|c|}{\textbf{Lugar:} ``Imagen y Comunicaci\'on S.A.''} \\
	\hline \multicolumn{3}{|p{0.86\linewidth}|}{\textbf{Presentes:} Por parte de la empresa: Daniela Paz (Legales y Personal)}  \\
	\multicolumn{3}{|p{0.86\linewidth}|}{Por parte de UBA:  Hurtado, Pablo; Zavala, Diego; Zhang, Yi Cheng} \\
	\hline
    \rowcolor[gray]{0.8} & & \\
    \hline
    \textbf{Tema} & \textbf{Solución} & \textbf{Responsable}\\
    \hline
    \multicolumn{1}{|p{0.33\linewidth}|}{Historia de la empresa} & \multicolumn{1}{p{0.33\linewidth}|}{Se nos coment\'o sobre la creaci\'on de la empresa, una breve historia de su creaci\'on y su evoluci\'on.} & \multicolumn{1}{p{0.20\linewidth}|}{Zavala, Diego}\\\hline
    \multicolumn{1}{|p{0.33\linewidth}|}{Organizaci\'on general de la empresa} & \multicolumn{1}{p{0.33\linewidth}|}{Nos brind\'o una explicaci\'on del funcionamiento de la empresa, las jerarqu\'ias en la misma, las divisiones de trabajo, y las funciones de cada gerente/jefe.} & \multicolumn{1}{p{0.20\linewidth}|}{Zhang, Yi Cheng}\\\hline
    \multicolumn{1}{|p{0.33\linewidth}|}{Recorrido por la f\'abrica} & \multicolumn{1}{p{0.33\linewidth}|}{Se realiz\'o un tour por todos los sectores de la f\'abrica para poder entender mejor su funcionamiento y c\'omo elaboran sus productos.} & \multicolumn{1}{p{0.20\linewidth}|}{Hurtado, Pablo}\\\hline
    \rowcolor[gray]{0.8} & & \\\hline
    \multicolumn{3}{|p{0.86\linewidth}|}{\textbf{Conclusiones:} En esta primera reuni\'on, adem\'as de sus caracter\'isticas e informaci\'on general, pudimos entender bien c\'omo funciona la empresa, y la funci\'on de cada uno de los gerentes y jefes.}\\\hline
    \multicolumn{3}{|p{0.86\linewidth}|}{\textbf{Distribuir a:} Presentes, Ing. Barmack, Enrique Paz, Grupo B1}\\\hline
\end{tabular} 
\end{center}

\vspace{6cm}
\textbf{Pasos a seguir:}\\
\begin{center}
\begin{tabular}{|c|c|c|}
	\hline \textbf{Tarea} & \textbf{Encargado} & \textbf{Fecha de entrega} \\ 
	\hline Resumen de la entrevista & Hurtado, Pablo & 06/04/11 \\ 
	\hline Organigrama & Hurtado, Pablo & 06/04/11 \\ 
	\hline 
\end{tabular}
\end{center}

\textbf{Tomaron nota:} Zavala, Diego; Zhang, Yi Cheng

\textbf{Digitalizaci\'on:} Hurtado, Pablo; Zhang, Yi Cheng



\newpage

\subsubsection*{Resumen de la reuni\'on}
A los cuatro d\'ias del mes de abril del a\~no 2011, en la sede de Imagen y Comunicaci\'on S.A., se re\'unen los alumnos con la Srta. Daniela Paz, empleada administrativa de la empresa.
Daniela trabaja all\'i hace 5 a\~nos, y si bien en el organigrama aparece como encargada de asuntos legales y de personal, tambi\'en realiza trabajos administrativos, de apoyo contable y de tesorer\'ia.

Se realiz\'o una serie de preguntas y se obtuvieron las respuestas detalladas a continuaci\'on.

\begin{itemize}

\item ¿Cu\'antos a\~nos tiene la empresa? \\
Se fund\'o en el 2000, as\'i que 11. La fecha exacta no supo decirnos.
\item ¿Cu\'al es el rubro de la misma? \\
Metal\'urgica.
\item ¿Con cu\'anto personal cuenta? \\
45 personas.
\item ¿Qu\'e operaci\'on tiene con otras empresas? \\
Tiene un contrato de exclusividad con Nike y cualquier particular que quiera poner un local que venda productos Nike 
en alguna parte de Latinoam\'erica (excepto Brasil y M\'exico) tiene que ponerse en contacto con Imagen y Comunicaci\'on S.A. 
para pedir mobiliario. No tienen otros clientes. Hasta hace poco tercerizaban el trabajo de diseño gr\'afico, pero ahora lo hacen ellos mismos.
\item ¿Hace cu\'anto que es proveedor de Nike? \\
Desde que la empresa se fund\'o.
\item ¿Con cu\'antas sucursales cuenta? \\
Ninguna. Aparte de la sede, tienen un galp\'on a una cuadra, pero no hay m\'as oficinas que las de la sede.
\item ¿Cu\'al es su pol\'itica con los trabajadores? \\
La jerarqu\'ia entre los trabajadores es puramente por antig\"uedad: los jefes de cada secci\'on son los trabajadores m\'as antiguos
 y experimentados de la misma. La comunicaci\'on es informal. No hay delegados sindicales en la empresa, aunque est\'an en la UOM.
\item ¿Existe alg\'un plan o deseo para proveer a otra empresa en el futuro? \\
S\'i existe el deseo, pero por el tamaño de la empresa ser\'ia imposible.
\item ¿Hay mucha competencia en el rubro? De ser s\'i, c\'omo la manejan? \\
En lo que hace a esta empresa en particular, no hay competencia de ning\'un tipo, dado que posee un contrato de exclusividad.
\item ¿Qu\'e productos nuevos se est\'an solicitando y deber\'an empezar a producir? (en caso de que elaboran productos a pedido de las empresas) \\
Est\'an sujetos a los cambios de l\'inea de Nike. Cuando cambia la l\'inea hay que empezar a producir muchos productos nuevos,
 para los cuales puede haber que usar t\'ecnicas diferentes a las que se ven\'ian utilizando. 
Por ejemplo, ha pasado que se ha tenido que reemplazar el sector de Acr\'ilico por un sector de Pintura por un cambio de l\'inea. 
Como los trabajadores no est\'an fuertemente especializados, se puede realizar el nuevo trabajo sin tener que hacer cambio de personal, pero con un costo de formaci\'on.
\item ¿La infraestructura (instalaciones) con la que cuentan alcanzan para el buen desarrollo del trabajo? \\
 A simple vista pareciera que no, pero si han logrado mantener un contrato de semejante exigencia durante m\'as de 11 años, debe ser suficiente.
\item ¿Cuentan con muchas tecnolog\'ias para el desarrollo de los productos o m\'as bien los productos son elaborados en forma ``artesanal''? \\
Cuentan con alta tecnolog\'ia. Tienden a renovar tecnolog\'ias en cuanto pueden (Le preguntamos si les hab\'ia pasado como en el caso Dolly; nos dijo que no, pero casi.) La figura de la cancha es la m\'aquina que corta metal con un chorro de agua con abrasivos, que tiene mayor precisi\'on que un l\'aser.
\item ¿Piensan introducir nuevas tecnolog\'ias? \\
Seguramente.
\item ¿Qui\'enes se encargan del diseño para la elaboraci\'on de los productos? Los diseños son propios o los da la empresa que pide el producto? \\
Claramente los pide Nike. Por ejemplo, una de las cosas que nos mostraron fue un escudo del Manchester Utd., y hab\'ia varios logos de Nike en las oficinas, 
y adem\'as, por la naturaleza de su trabajo (exclusivamente mobiliario de negocios deportivos), tienen que representar fielmente \'iconos que al p\'ublico le resulten 
representativos de las marcas, de modo que mucha libertad no tienen.
\item ¿Para la fabricaci\'on de sus productos, cuentan con proveedores que les venden la materia prima o se autoproveen de la misma? \\
Cuentan con proveedores. 
\end{itemize}

\subsubsection*{\textbf{Organigrama Tentativo}}
\begin {center}
	\includegraphics[angle=90,scale=0.55]{./IcomOrganigramaNuestro1.png}
\end{center}

\newpage
\subsection{Minuta de Reuni\'on del d\'ia 25 de abril}
\vspace{1cm}
\begin{center}
\begin{tabular}{|c|c|c|}
	\hline
	\textbf{Fecha:} 25/04/2011 &  \textbf{Comienzo:} 15:00  &  \textbf{Fin:} 16:00 \\ \hline	

	\multicolumn{3}{|c|}{\textbf{Lugar:} ``Imagen y Comunicaci\'on S.A.''} \\
	\hline \multicolumn{3}{|p{0.86\linewidth}|}{\textbf{Presentes:} Por parte de la empresa: Enrique Paz (Director General)} \\
	\multicolumn{3}{|p{0.86\linewidth}|}{Por parte de UBA: Aberastury, Luc\'ia; Barbieri, Alejandro; Villanueva, Amalia; Camacho, Tania Gabriela; Garbarini, Luc\'ia} \\
	\hline
    \rowcolor[gray]{0.8} & & \\
    \hline
    \textbf{Tema} & \textbf{Solución} & \textbf{Responsable}\\
    \hline
    \multicolumn{1}{|p{0.33\linewidth}|}{Conocimiento de detalles sobre la historia de la empresa.} & \multicolumn{1}{p{0.33\linewidth}|}{Se habló con el Director General y fundador de la empresa sobre la historia de la misma.} & \multicolumn{1}{p{0.20\linewidth}|}{Aberastury, Luc\'ia}\\\hline
    \multicolumn{1}{|p{0.33\linewidth}|}{Profundizaci\'on en el tema de organizaci\'on general de la empresa y las funciones de cada gerente.} & \multicolumn{1}{p{0.33\linewidth}|}{Se consult\'o al presente por parte de la empresa acerca de sus funciones y las de sus subordinados.} & \multicolumn{1}{p{0.20\linewidth}|}{Barbieri, Alejandro}\\\hline
    \multicolumn{1}{|p{0.33\linewidth}|}{Relaci\'on con el cliente (Nike), y relaciones formales e informales dentro de la empresa.} & \multicolumn{1}{p{0.33\linewidth}|}{Se habl\'o con el presente por parte de la empresa acerca de la comunicaci\'on interna y externa, y sobre casos anecd\'oticos que marquen las relaciones bajo an\'alisis.} & \multicolumn{1}{p{0.20\linewidth}|}{Villanueva, Amalia}\\\hline
    \rowcolor[gray]{0.8} & & \\\hline
    \multicolumn{3}{|p{0.86\linewidth}|}{\textbf{Conclusiones:} Se conoci\'o la historia de la empresa contada por su fundador. Se comprendi\'o que las relaciones informales en la empresa prevalecen sobre las formales.}\\\hline
    \multicolumn{3}{|p{0.86\linewidth}|}{\textbf{Distribuir a:} Presentes, Ing. Barmack, Enrique Paz, Grupo B1}\\\hline
\end{tabular} 
\end{center}

%\vspace{4cm}
%\textbf{S\'intesis de los temas tratados:}

%\begin{enumerate}
%	\item Detalles sobre la historia de la empresa.
%
%	\item Profundizaci\'on en el tema de organizaci\'on general de la empresa y las funciones de cada gerente.
%
%   \item Relaci\'on con el cliente (Nike), y relaciones formales e informales dentro de la empresa.
%\end{enumerate}

\vspace{8cm}
\textbf{Pasos a seguir:}\\
	    \begin{center}
	    \begin{tabular}{|c|c|c|}
		    \hline \textbf{Tarea} & \textbf{Encargado} & \textbf{Fecha de entrega} \\ 
		    \hline Resumen de la entrevista & Garbarini, Luc\'ia & 27/04/11 \\ 
		    \hline Organigrama & Camacho, Tania Gabriela & 27/04/11 \\ 
		    \hline 
	    \end{tabular}
	    \end{center}

\newpage
\subsubsection*{Resumen de la reuni\'on}
La reunión comenzó con un racconto del origen de la empresa Imagen y Comunicaci\'on S.A., el cual se resume a continuaci\'on.


\textit{Hacia el año 1992, Enrique Paz diseñaba exhibidores de cart\'on con apliques de metal. 
Ese año hubo concursos para Nike y Reebok en los que present\'o exhibidores, presentando el mejor diseño (con un estilo m\'as cl\'asico) ante Reebok y el peor ante Nike, dado que Nike era una empresa relativamente nueva y que Reebok ten\'ia una imagen m\'as acorde al mejor de aquellos dos diseños.
El gerente de Nike en Argentina dijo que se comunicar\'ia con el señor Paz en 60 d\'ias, lo cual Enrique entendi\'o como un rechazo; sin embargo, aproximadamente dos meses despu\'es, lleg\'o un llamado en el cual le ped\'ian 300 exhibidores ``Pampa'' (el nombre original del producto, elegido por Enrique, era ``Black Black'', nombre originado por el color negro del material al trabajarlo, pero Nike decidi\'o cambiarlo sin previo aviso).
Con pocos recursos econ\'omicos y con proveedores amigos, se pudo cumplir el pedido.}


\textit{M\'as adelante, para otro concurso de Nike, Enrique present\'o un nuevo diseño llamado ``Canoa''.
Mientras que las otras dos empresas competidoras realizaron sus presentaciones con dibujos realizados por computadora, Paz entreg\'o dibujos hechos a mano por un muy buen artista amigo, lo cual a la gente de Nike le agrad\'o mucho, otorg\'andole la licencia a Enrique.
As\'i, la empresa unipersonal de Enrique Paz se fue convirtiendo en un proveedor de Nike y creciendo, al mismo tiempo que daba trabajo a sus proveedores (de materiales, pintura, etc.), que antes de todo esto no estaban pasando por un buen momento.
Un tiempo despu\'es, Paz propone fabricar no s\'olo los exhibidores para Nike, sino todo el mobiliario necesario para cualquier negocio que desee vender productos de esa marca, haciendo que Nike dependa de este proveedor m\'as fuertemente, de modo de reducir el riesgo inherente a trabajar para un \'unico cliente.}

\textit{Posteriormente se da un hecho muy importante en la relaci\'on con Nike, cuando los distintos proveedores de mobiliario de Nike en el mundo dan ideas para una nueva l\'inea de productos.
El proveedor canadiense propuso una l\'inea cromada, ante lo cual Enrique objet\'o que eso traer\'ia much\'isimos problemas de contaminaci\'on, sobre todo a la hora de desechar los productos cuando se acabara su vida \'util o se produjera un cambio de l\'inea (adem\'as de que, por la ley argentina, probablemente le hubiesen negado los permisos para producir en esas condiciones), y propuso en cambio el uso de acero inoxidable.
Finalmente fue aceptada la propuesta del señor Paz, dado que Nike vio que los problemas citados por el mismo eran factibles. Esto fortaleci\'o los v\'inculos entre las empresas. En el año 2000 finaliza la etapa como empresa unipersonal y se funda Imagen y Comunicaci\'on S.A.}

A pesar de los aspectos positivos de la relaci\'on entre la empresa y Nike, no hay comunicaci\'on constante entre las empresas, ya que lo que hace ``iCom'' (como le llaman coloquialmente a Imagen y Comunicaci\'on S.A.) es proveer de mobiliario a quienes quieran vender productos Nike en su negocio. Tampoco hay di\'alogo con estos clientes locales, sino que iCom s\'olo se limita a responder al pedido.

En cuanto a la organizaci\'on de la empresa, se puso en claro que en el sector de Administraci\'on los roles no est\'an bien definidos, siendo que a veces la encargada de Personal se encarga de trabajos de contadur\'ia (ya nos lo hab\'ia contado ella misma) e incluso la recepcionista hace trabajos administrativos. 

Enrique Paz es el Director General de la empresa, pero adem\'as se encarga del \'area de Ingenier\'ia y Desarrollo y de algunos asuntos administrativos. Como gerente, su objetivo principal es coordinar y controlar el trabajo del Ingeniero y la Arquitecta, y supervisa adem\'as a la gerente de Producci\'on. 

El sector de Ventas est\'a formado por \'unicamente dos personas: una se encarga de las ventas locales y una de todo asunto de comercio exterior (no s\'olo ventas). Una diferencia entre las ventas locales y las internacionales es que los clientes locales tienen los estudios de arquitectura necesarios realizados previamente, mientras que en el exterior hay que empezar todo de cero. La encargada del comercio exterior además actúa como gerente del departamento, supervisando al encargado de ventas locales.

El sector de Dise\~no Gr\'afico tiene la particularidad de que tiene sus clientes propios y se maneja de forma independiente del resto de la empresa (salvo de la Direcci\'on).

Ilce Pesaresi es la gerente de Producci\'on, el \'area m\'as grande de la empresa. Su funci\'on es asegurarse de que los empleados de planta trabajen correcta y eficientemente, y responde a Enrique Paz. 

Se observa que existe un predominio de las relaciones informales dentro de la empresa. El liderazgo en cada sector del n\'ucleo operativo lo toma el trabajador con mayor experiencia, casi \textit{de facto}. No se comunica a los obreros la ideolog\'ia ni la misi\'on de la empresa, ni ning\'un aspecto administrativo, de modo que es frecuente la aparici\'on de rumores y de dudas sobre su continuidad en la empresa.

\subsubsection*{\textbf{Organigrama Tentativo}}
\begin {center}
	\includegraphics[angle=90,scale=0.45]{./IcomOrganigramaNuestro2b.png}
\end{center}
\newpage

\subsection{Minuta de Reuni\'on del d\'ia 06 de mayo}
\vspace{1cm}
\begin{center}
\begin{tabular}{|c|c|c|}
	\hline
	\textbf{Fecha:} 06/05/2011 &  \textbf{Comienzo:} 15:00  &  \textbf{Fin:} 15:15 \\ \hline	

	\multicolumn{3}{|c|}{\textbf{Lugar:} ``Imagen y Comunicaci\'on S.A.''} \\
	\hline \multicolumn{3}{|p{0.86\linewidth}|}{\textbf{Presentes:} Por parte de la empresa: Ana Paula Benedetti (Gerente de Ventas)} \\
	\multicolumn{3}{|p{0.86\linewidth}|}{Por parte de UBA: Garbiso, Julian; Zurita, Stephanie; Ygounet, Guido } \\
	\hline
    \rowcolor[gray]{0.8} & & \\
    \hline
    \textbf{Tema} & \textbf{Solución} & \textbf{Responsable}\\
    \hline
    \multicolumn{1}{|p{0.33\linewidth}|}{Conocimiento de la estructura de \'area de ventas.} & \multicolumn{1}{p{0.33\linewidth}|}{Se habl\'o con la Gerente de Ventas sobre la conformaci\'on de dicho \'area.} & \multicolumn{1}{p{0.20\linewidth}|}{Stephanie, Zurita}\\\hline
    \multicolumn{1}{|p{0.33\linewidth}|}{Profundizaci\'on sobre la organizaci\'on de dicha \'area y las responsabilidades de cada gerente dentro del mismo.} & \multicolumn{1}{p{0.33\linewidth}|}{Se le consult\'o al respecto de las responsabilidades de cada uno de los gerentes de dicho \'area y sus funciones. } & \multicolumn{1}{p{0.20\linewidth}|}{Garbiso, Julian}\\\hline
    \rowcolor[gray]{0.8} & & \\\hline
    \multicolumn{3}{|p{0.86\linewidth}|}{\textbf{Conclusiones:} Se conoci\'o de forma mas profunda la organizaci\'on del departamento de ventas de la empresa, as\'i como tambi\'en las responsabilidades y funciones de quienes lo conforman.}\\\hline
      \multicolumn{3}{|p{0.86\linewidth}|}{\textbf{Distribuir a:} Presentes, Ing. Barmack, Ana Paula Benedetti, Grupo B1}\\\hline
\end{tabular} 
\end{center}

%\vspace{4cm}
%\textbf{S\'intesis de los temas tratados:}

%\begin{enumerate}
%	\item Detalles sobre la historia de la empresa.
%
%	\item Profundizaci\'on en el tema de organizaci\'on general de la empresa y las funciones de cada gerente.
%
%   \item Relaci\'on con el cliente (Nike), y relaciones formales e informales dentro de la empresa.
%\end{enumerate}

\vspace{10cm}
\textbf{Pasos a seguir:}\\
	    \begin{center}
	    \begin{tabular}{|c|c|c|}
		    \hline \textbf{Tarea} & \textbf{Encargado} & \textbf{Fecha de entrega} \\ 
		    \hline Resumen de la entrevista & Ygounet, Guido Nahuel & 07/05/11 \\ 
		    \hline Organigrama & Garbiso, Julian & 10/05/11 \\ 
		    \hline 
	    \end{tabular}
	    \end{center}

\newpage
\subsubsection*{Resumen de la reuni\'on}

La Srta. Benedetti nos coment\'o acerca de como est\'a conformado el \'area que se encuentra a su cargo. La reuni\'on debi\'o desarrollarse r\'apidamente debido a que la persona en cuesti\'on se encontraba sumamente ocupada; pero as\'i y todo nos brind\'o algunos minutos de su tiempo.
\\ Nos coment\'o que el \'area de ventas est\'a dividida en dos sectores: El sector de Ventas Locales y el sector de Comercio Exterior, tal cu\'al nos hab\'ia comentado el Sr. Paz en nuestra entrevista anterior.
Al respecto de dichos sectores nos cont\'o que en este \'area son b\'asicamente dos personas; ella misma, que es la encargada del sector de Comercio Exterior y la Srta. Ver\'onica  Yelli, encargada del sector de Ventas Locales.
\\ De igual modo, coment\'o que las relaciones entre ambos miembros del sector se dan generalmente por la v\'ia informal, y que no exist\'ia (dado que nunca lo hab\'ia usado) un canal de comunicaci\'on formal o alg\'un proceso estandarizado que debiera seguir.
\\ Tambi\'en rescatamos la diferencia que subray\'o entre ambos sectores. La misma es que el sector de Ventas Locales trabaja con estudios de arquitectura realizados previamente para cada uno de sus clientes. Mientras que, en el sector del que ella se encarga(Comercio Exterior), se realizan
los estudios de arquitectura desde cero para cada uno de los clientes. Esto de alguna manera, marca una diferencia en los plazos de entrega a unos y otros clientes de cada uno de los sectores.
\\ En cuanto a las responsabilidades y funciones nos coment\'o r\'apidamente que: ella era responsable de encargarse de las ventas que se realicen a todo cliente extranjero por estar a cargo de Comercio Exterior. Que la Srta. Yelli se encargaba de igual forma de las ventas a clientes locales.
Y que su responsabilidad y funci\'on como gerente del \'area de ventas era controlar que la Srta. Yelli cumpliera con su trabajo e intentar solucionar los posibles inconvenientes que surgieran tanto con ventas a clientes locales como internacionales. Sobre esto \'ultimo nos remarc\'o que ella
hac\'ia un esfuerzo por no priorizar a un cliente sobre otro y por intentar tratar a los clientes de ambos sectores por igual (sin darle importancia a que los internacionales dependieran directamente de ella, mientras que los locales lo hicieran indirectamente).

\subsubsection*{\textbf{Organigrama derivado de la reuni\'on}}
\begin{center}
	\includegraphics[angle=90,scale=0.45]{./IcomOrganigramaNuestro3.png}
\end{center}



\newpage

% Comando \puesto. Uso: \puesto{Nombre del puesto}{Dirección}{Responde a}{Descripción del puesto}{Funciones}{Responsabilidad}{Autoridad}
\newcommand{\puesto}[7]{
\begin{tabular}{|c|l|}\hline
& Direcci\'on: #2 \\\cline{2-2}
& Responde a: #3 \\\cline{2-2}
& \begin{minipage}{0.9\textwidth}
    \vspace{0.5cm}
    \textit{Descripci\'on del puesto}:
    #4
    \vspace{0.5cm}
\end{minipage}
\\
\cline{2-2}
\multirow{4}{*}{\begin{sideways}{\textbf{\large#1}}\end{sideways} } & \begin{minipage}{0.9\textwidth}
    \vspace{0.5cm}
    \textit{\textbf{Funciones}}:
    #5
    \vspace{0.5cm}
\end{minipage}
\\
\cline{2-2}
& \begin{minipage}{0.9\textwidth}
    \vspace{0.5cm}
    \textit{\textbf{Responsabilidad}}:
    #6
    \vspace{0.5cm}
\end{minipage} 
\\
\cline{2-2}
& \begin{minipage}{0.9\textwidth}
    \vspace{0.5cm}
    \textit{\textbf{Autoridad}}:
    #7
    \vspace{0.5cm}
\end{minipage}\\\hline
%& \begin{minipage}{0.9\textwidth}
%    \vspace{0.5cm}
%    \textit{Perfil del Puesto}:
%    #8
%    \vspace{0.5cm}
%\end{minipage}\\\hline
\end{tabular}

}

\subsection{ Manual de Funciones }

\subsubsection{Direcci\'on General}

\puesto{Direcci\'on General}{Direcci\'on}{Nadie}
{
    Gerente General
}
{
 \begin{itemize}
    \item[-] Liderar el planeamiento estrat\'egico de la empresa
    \item[-] Desarrollar estrategias generales para alcanzar estos objetivos propuestos
    \item[-] Distribuir estos objetivos entre las diferentes \'areas
    \item[-] Seleccionar personal competente para liderar los diferentes departamentos
    \item[-] Aprobar, modificar o rechazar todo tipo de sugerencias
 \end{itemize}
}
{
 \begin{itemize}
    \item[-] Planear, dirigir y controlar las actividades de la empresa
    \item[-] Fijar los objetivos generales (globales) de la empresa
    \item[-] Asignar objetivos y recursos a cada sector y departamento
    \item[-] Velar por el buen funcionamiento de la empresa
 \end{itemize}
}
{ 
 \begin{itemize}
    \item[-] Autoridad absoluta sobre todo el personal de la empresa
    \item[-] Puede delegar y cambiar autoridades, funciones y responsabilidades a los gerentes de los diferentes departamentos
 \end{itemize}
}
%{

%}

\newpage

\subsubsection{Dise\~no Gr\'afico}
\smallskip
Del relevamiento de este sector surgen diferencias respecto del organigrama presentado por la empresa, ya que si bien se ocupa de las impresiones y dise\~nos de la misma, 
tambi\'en realiza trabajos para otras empresas, por lo que puede decir que no cumple funciones de Linea sino de Staff.
\smallskip

\puesto{Dise\~no gr\'afico}{Dise\~no gr\'afico}{Gerente General}
{
    Dise\~nador Gr\'afico
}
{
    \begin{itemize}
        \item[-] Coordinar los dise\~nos a producir con el cliente.
        \item[-] Realizar los dise\~nos pedidos.
        \item[-] Realizar impresiones internas de la empresa y Plotters.
        \item[-] Colocar Plotters.
    \end{itemize}
}
{
    \begin{itemize}
        \item[-] Asegurar que los dise\~nos producidos sean de la calidad necesaria.
        \item[-] Cumplir las pautas establecidas por los clientes.
        \item[-] Informar a la direcci\'on acerca de los trabajos realizados y las necesidades de recursos.
    \end{itemize}
}
{
    No tiene, por cumplir funciones de Staff.
}
%{
%    \begin{itemize}
%      \item Apreciaci\'on est\'etica, imaginaci\'on y originalidad, que permitan el desarrollo de proyectos novedosos, creativos y de valor pl\'astico.
%      \item Agudeza visual, agilidad manual y precisi\'on para el manejo de equipo y herramientas, para la aplicaci\'on de procesos t\'ecnicos y art\'isticos.
%      \item Habilidad para dise\~nar y dibujar. 
%      \item Habilidad para conceptualizar y desarrollar proyectos a partir de la recopilaci\'on, la interpretaci\'on y el an\'alisis de la informaci\'on. 
%      \item Facilidad de palabra y sentido cr\'itico que permitan tomar, analizar, explicar y justificar decisiones de toda \'indole.
%    \end{itemize}
%}
\newpage
\subsubsection{Gerencia de Administraci\'on de Personal y Finanzas}

 \begin{itemize}
   \item Administraci\'on contable
   \item Legales y personal
   \item Recepci\'on
 \end{itemize}
 En este sector se notan importantes diferencias respecto del organigrama relevado por la empresa; las funciones en el mismo no est\'an correctamente definidas ya que la encargada de personal se ocupa adem\'as de trabajos de contaduría y la recepcionista realiza trabajos administrativos.
 \bigskip
 \bigskip
 \puesto{Administraci\'on Contable}{Administraci\'on del personal y finanzas}{Direcci\'on General}{Encargado de Contabilidad y Finanzas}
 {
    \begin{itemize}
        \item[-] Gestionar y administrar el dinero
        \item[-] Hacer los balances de la empresa
        \item[-] Efectuar estimaciones de gastos
    \end{itemize}
 }
 {
    \begin{itemize}
        \item[-] Gestionar los cr\'editos y las cobranzas
        \item[-] Realizar los pagos a proveedores, conciliaciones bancarias
    \end{itemize}
 }
 {
    \begin{itemize}
         \item[-] Determinar incobrables
         \item[-] Emitir balance anual de la empresa, libros de IVA
    \end{itemize}
 }
 %{
 
% }

\newpage

\puesto{Legales y Personal}{Administraci\'on del personal y finanzas}{Direcci\'on General}{Administraci\'on del Personal y Legales}
{
    \begin{itemize}
        \item[-] Incorporaci\'on y capacitaci\'on del personal
        \item[-] Coordinaci\'on y control de liquidaci\'on de sueldos, c\'alculo de impuesto a las ganancias, deducciones
        \item[-] Evaluaci\'on de las actividades vigilando la observancia de leyes y reglamentos aplicables
        \item[-] Dem\'as tareas administrativas relaccionadas con recursos humanos: n\'ominas, vacaciones, pr\'estamos, etc.
        \item[-] Apoyo contable y tesorer\'ia
    \end{itemize}
}
{
    \begin{itemize}
         \item[-] Controlar y llevar a cabo la liquidaci\'on de sueldos en forma mensual, coordinar vacaciones, cuidar que todos los empleados tengan garantizado un respaldo m\'edico
        \item[-] Gesti\'on del desempeño, capacitaci\'on de nuevos empleados
        \item[-] Mantenerse actualizado y cumplir con las normativas vigentes
        \item[-] Generar un ambiente de trabajo que permita el desarrollo personal y profesional
    \end{itemize}
}
{
    \begin{itemize}
        \item[-] Aceptar, solicitar modificaciones o rechazar pedidos de modificaci\'on de sueldo
        \item[-] Denunciar incumplimientos en las normativas laborales
    \end{itemize}
}
%{

%}

\newpage

\puesto{Recepci\'on}{Administraci\'on del personal y finanzas}{Legales y Personal}{Recepci\'on}
{
    \begin{itemize}
        \item[-] Ingreso de visitas
        \item[-] Trabajos administrativos
    \end{itemize}
}
{
    \begin{itemize}
        \item[-] Correspondencia
        \item[-] Agenda de gerentes
    \end{itemize}
}
{

}
%{

%}

\newpage

\subsubsection{Gerencia de Comercializaci\'on}

En este sector se realiza el contacto con el cliente, desde la asesor\'ia al cliente de los dise\~nos de arquitectura 
hasta la concretaci\'on de la venta. Est\'a formado por dos sectores, uno que se encarga de las ventas locales y otra
de las ventas en el exterior.\\
El trato con los clientes es escaso ya que no se lleva a cabo ning\'un tipo de estrategia comunicacional. La confianza 
en la empresa est\'a basada en la antig\"uedad del contrato con Nike. Sin embargo, junto con la Direcci\'on de 
la empresa, mantiene un fuerte v\'inculo el presidente de Nike Argentina SRL, su gerente de finanzas y su gerente de retail.
\smallskip

\puesto{Sector de Ventas Locales}{Gerencia de Comercializaci\'on}{Direcci\'on General}{ Responsable de Ventas Locales}
{
    \begin{itemize}
	\item[-] Recibir y Coordinar Pagos
	\item[-] Contactar clientes locales e informar modificaciones del cat\'alogo de productos. 
	\item[-] Coordinar plazos de entrega
	\item[-] Controlar transacciones con clientes nacionales.
    \end{itemize}
}
{
    \begin{itemize}
	\item[-] Priorizar a los clientes mayoristas y fortalecer la comunicaci\'on con ellos.
	\item[-] Asegurar el cumplimiento de plazos de entrega convenidos
	\item[-] Contactar clientes para sugerir nuevas compras.
    \end{itemize}
}
{
      \begin{itemize}
	\item[-] Informar a producci\'on sobre los requerimientos de productos
	\item[-] Exigir cumplimiento de plazos de entrega a producci\'on
      \end{itemize}
}
%{

%}

\puesto{Sector de Comercio Exterior}{Gerencia de Comercializaci\'on}{Direcci\'on General}{ Responsable de Comercio Exterior}
{
    \begin{itemize}
        \item[-] Recibir y Coordinar Pagos con los clientes del exterior
        \item[-] Contactar y coordinar plazos de entrega y precios.
        \item[-] Planificar pedidos de apertura y remodelaci\'on de locales.
        \item[-] Controlar transacciones con clientes internacionales.
    \end{itemize}
}
{
    \begin{itemize}
        \item[-] Mantener contacto estrecho con los clientes del exterior.
        \item[-] Informar a los clientes el cat\'alogo de productos
        \item[-] Asesorar a los clientes al momento de la realizaci\'on de un pedido.
    \end{itemize}
}
{
    \begin{itemize}
        \item[-] Informar a Producci\'on sobre los requerimientos de productos
        \item[-] Exigir cumplimiento de plazos de entrega a producci\'on
        \item[-] Determinar, junto al gerente general, las pol\'iticas exteriores.
    \end{itemize}
}
%{

%}

\newpage

\subsubsection{Gerencia de Ingenier\'ia y Desarrollo}
\smallskip
Este sector se encarga de prestar los medios materiales necesarios para desarrollar las metas que se propone la empresa. 
Su acci\'on conjunta con Planeamiento es de gran importancia en t\'erminos de innovaci\'on para los productos y respuestas inmediatas a las exigencias 
del mercado en general y de su principal cliente, Nike.

\puesto{Ingenier\'ia y Desarrollo}{Gerencia de Ingenier\'ia y Desarrollo}{Direcci\'on General}
{
Comprende todo lo relacionado con la investigaci\'on b\'asica y aplicada y el desarrollo para el mejoramiento de productos, 
procesos nuevos e investigaci\'on para el mejoramiento de procesos y desarrollo.
}
{
\begin{itemize}
	\item[-] Su principal inter\'es se centra en la optimizaci\'on de los recursos para su mejor aprovechamiento.
		Si bien los procesos de industrializaci\'on en la empresa han sido siempre paulatinos, 
		esta \'area tiene la funci\'on de suministrar los recursos t\'ecnicos y formales necesarios para que el empleado
		pueda desarrollarse adecuadamente en el \'ambito laboral en el que est\'a inserto, atendiendo a sus necesidades 
		de modo de lograr un equilibrio entre las satisfacci\'on laboral de su personal y las exigencias del mercado.
	
\end{itemize}
}
{
\begin{itemize}
   \item[-] Implementar sistemas de producci\'on m\'as eficientes y funcionales al sistema que respeten los valores intr\'insecos de la empresa. 
	    En ese sentido, no s\'olo tiene la responsabilidad de llevar al cabo los distintos planes productivos que muchas veces traen aparejados 
	    mejoras tecnol\'ogicas sino tambi\'en mejorar las condiciones de producci\'on para alcanzar as\'i una mayor efectividad. 
   
   \item[-] Ofrecer respaldo a las distintas demandas de los empleados y promover iniciativas que mejoren su calidad de trabajo. 
   
   \item[-] Mantener en curso los proyectos de capacitaci\'on en caso de que sean necesarios. 
   
   \item[-] Abastecer a la estructura con la mejor calidad de infraestructura y ser capaz de contrarrestar  las flaquezas por falta de inversi\'on, a trav\'es de otros recursos. 
   
   \item[-] Llevar al cabo las medidas haciendo observar los valores que identifican a la empresa. 

\end{itemize}	
}
{
\begin{itemize}

   \item[-] Cumplimiento de la jornada laboral. 
   
   \item[-] Utilizaci\'on responsable de las tecnolog\'ias adquiridas por la empresa. 
   
   \item[-] Emplear el nuevo software especialmente dise\~nado para que la gerencia se ponga en contacto con el personal y as\'i lograr una comunicaci\'on a trav\'es de la cual el empleado sienta que sus demandas sugerencias o propuestas son tenidas en cuenta 

\end{itemize}
}
%{

%}

\newpage

\subsubsection{Gerencia de Producci\'on}
\smallskip
En este sector incluye las operaciones de complejidad productiva: coordina y regula la manufactura de bienes fabriles y sistematiza el mantenimiento de los distintos módulos que integran el proceso de producción.

\puesto{Gerencia de producción}{Gerencia de Ingenier\'ia y Desarrollo}{Direcci\'on General}
{


}
{
\begin{itemize}
	\item[-] Su función en principio es la de guiar y mantener constantes los volúmenes de producción establecidos. 

	\item[-] En una segunda línea, también debe llevar al cabo todos los proyectos aprobados por la gerencia que aumenten la producción cuantitativa y cualitativa.  

	\item[-]A fin de optimizar los márgenes de utilidad, también debe estar con comunicación constante con los demás departamentos que dependan de su acción. 
	
\end{itemize}
}
{
\begin{itemize}
   \item[-] Asegurar los insumos necesarios para la producción asimismo como los medios tecnológicos propicios. 
   
   \item[-] Fomentar la optimización de los bienes que ya posee la empresa (tanto capital humano, como maquinaria) a fin de reducir los costos y generar más ganancias. 
   
\end{itemize}	
}
{
\begin{itemize}

   \item[-] Hacer cumplimentar las normas de producción establecidas por la empresa, siempre en concordancia con los valores propuestos por la empresa. 
   
   \item[-] Intervenir en los procesos productivos a modo de reportar luego posibles inconvenientes o deficiencias que deban solucionarse a futuro. 
   
   \item[-] En vista de alcanzar un mayor rendimiento ejercer una política efectiva de presión hacia los empleados. Cumplimiento de la jornada laboral. 

\end{itemize}
}
%{

%}

\newpage
\subsection{Hipótesis}
\vspace{1cm}
\indent \textit{Imagen y Comunicación debido a la calidad demostrada hasta la fecha recibe una propuesta de la casa matriz de Nike a los efectos de ocuparse de todo lo relacionado con Imagen y Comunicación en Argentina en la instalación de 200 locales de 2500 metros cuadrados cada uno en diferentes provincias del país. ¿Cómo adecuaría la estructura de la empresa para atender esta solicitud a lo largo de 5 años? Siendo que los diseños de los productos deben ser novedosos.}

\newpage
\vspace*{\fill}
\begin{center}
\begingroup
\titlerule
\vspace{1cm}
\section{Presentaci\'on}
\vspace{1cm}
\titlerule
\endgroup
\end{center}
\vspace*{\fill}


\end{document}
